\documentclass[12pt]{article}
\usepackage[margin=1in]{geometry}
\usepackage{amsmath}
\usepackage{amssymb}
\usepackage{fancyhdr}
\usepackage{pgfplots} 

%opening
\title{Linear Algebra Review}
\author{Steven Koniaev}

\pagestyle{fancy}
\renewcommand{\headrulewidth}{0pt}
\renewcommand{\footrulewidth}{0pt}



\fancyhf{}
\rhead{
	Steven K \\
	Prof: Me \\
	Math 133
}
\rfoot{Page \thepage}


\begin{document}
	\begin{center}
		Linear Algebra Notes
	\end{center}
	
	\section*{Study Sheets}
	
	\begin{flushleft}
		\textbf{Part One:} \\
		\begin{itemize}
			\item \textbf{Linear Equation: } \\ Equation in the form $a_1x_1+a_2x_2 + \ldots a_nx_n = b$
			\item \textbf{Solution: } \\ Parallel lines - No Solution - Inconsistent 
			\\ Same line - Infinite Solutions - Consistent \\ Intersection - One Solution - Consistent \\ This concept extends to 3D plans or even higher dimensions \\Number of variables = dimension of space
		\item \textbf{Matrix Notation: }
				$$ 7x + 5y - 3z = 16$$
				$$3x-5y + 2z = -8 $$
				$$5x + 3y - 7z = 0 $$
				
				\begin{center}
					Coefficient matrix
				$$\begin{bmatrix}
					7 & 5 & -3 \\
					3 & -5 & 2 \\
					5 & 3 & -7 \\
				\end{bmatrix}$$
			
			
				
					Augmented matrix
				$$\begin{bmatrix}
					7 & 5 & -3  & 16\\
					3 & -5 & 2 & -8\\
					5 & 3 & -7 & 0\\
				\end{bmatrix}$$
			m equations and n variables leads to a m x n matrix.
			
		\end{center}
	
			\item \textbf{Elementary Row Operations}\\
			\begin{itemize}
				\item[-]Row Swap) Exchange any two rows.
				\item[-](Scalar Multiplication) Multiply any row by a constant.
				\item[-](Row Sum) Add a multiple of one row to another row
			\end{itemize}
		
		\newpage
			\item \textbf{Reduced Row-Echelon Form}\\
			Example:
			
			$$\begin{bmatrix}
				1 & 0 & 0  & a\\
				0 & 1 & 0 & b\\
				0 & 0 & 1 & c\\
			\end{bmatrix}$$
			
			\begin{itemize}
				\item[-] Any row with non-zero entries has a leading one
				\item[-] All entries in a column with a leading one are zero
				\item[-] Each leading one has another up and to the left
			\end{itemize}
		Elimination is complete and we can look at the Rank of the matrix
	
	\begin{center}
		\framebox[1.1\width] {Rank = number of leading ones} \par 
	\end{center}

\begin{itemize}
	\item[-] Rank = Number of columns in coefficient matrix $\rightarrow$ at most one solution
	\item[-] Rank $<$ Number of columns in coefficient matrix $\rightarrow$ Infinite or No solutions
\end{itemize}
		
		
\item \textbf{Types of Matrices}
\begin{center}
	 m = number of rows\\
	 n = number of columns \\
	 
	 m = n results in a square matrix\\
\end{center}			
\begin{flushleft}
	The identity matrix I is an RREF square matrix with any size. \\
	Example:\\ 
	$$ I =
	\begin{bmatrix}
		1 & 0 & 0 & 0 \\ 
		0 & 1 & 0 & 0 \\
		0 & 0 & 1 & 0 \\
		0 & 0 & 0 & 1
	\end{bmatrix}
	$$
	\begin{itemize}
		\item[-] An upper triangular matrix is one where all entries below the diagonal are 0.
		\item[-] A lower triangular matrix one where all entries above the diagonal are 0.
		\item[-]  \textbf{Vector: }Matrix of one column
		\item[-]  \textbf{Row Vector: }Matrix of one row
	\end{itemize}
Linear equations can also be represented as vectors.
$$ 3x+y = 7$$
$$x+2y = 4 $$
$$ x \begin{bmatrix}3\\1\\ \end{bmatrix}
	+ y\begin{bmatrix}1\\2\\ \end{bmatrix} 
	= \begin{bmatrix}
		7 \\ 4 
	\end{bmatrix}$$
	
	This is the vector form of the linear system.
	This shows that scalars can be factored in and out of Matrices.
	
	\begin{itemize}
		\item[-] \textbf{Matrix Addition}\\
		Two Matrices must have the same dimension in order to add them.\\
		Example:
		
		$$\begin{bmatrix}
			1 & 2  & 3 \\
			4 & 5 & 6 \\
		\end{bmatrix}
	+  \begin{bmatrix}
		7 & 3  & 1 \\
		5 & 3 & -1\\
	\end{bmatrix}
= \begin{bmatrix}
	8 & 5  & 4 \\
	9 & 8 & 5 \\
\end{bmatrix} $$
Reminder that this is a 2 x 3 matrix and is the sum of the first two.\\
\item[-] \textbf{Matrix Subtraction} \\ Works in the exact same way as addition. \\

\textit{Note* \\Addition is commutative- Subtraction is not}\\

\item[-] \textbf{Matrix Multiplication}\\
In order for two matrices (n x m and q x p) to be multiplied. \\
m = q MUST be true. \\



	\end{itemize}
\end{flushleft}
		
		
		
		\end{itemize}
	\end{flushleft}
	
	
	\paragraph{1.1}
	\textit{
		\textbf{A}	Find this this this and that!
	}
\end{document}
