\documentclass[12pt]{article}
\usepackage[margin=1in]{geometry}
\usepackage{amsmath}
\usepackage{amssymb}
\usepackage{fancyhdr}
\usepackage{pgfplots} 

%opening
\title{Probability Notes}
\author{Steven Koniaev}

\pagestyle{fancy}
\renewcommand{\headrulewidth}{0pt}
\renewcommand{\footrulewidth}{0pt}



\fancyhf{}
\rhead{
	Steven K \\
	Prof: Jerome \\
	Math 240
}
\rfoot{Page \thepage}


\begin{document}
	
	\definecolor{light-gray}{gray}{0.95}
	\newcommand{\code}[1]{\colorbox{light-gray}{\texttt{#1}}}. 
	
	\begin{center}
		Discrete Structures Notes
	\end{center}

\section* {Sets} 
\underline{\textbf{Defining Sets}}
	\begin{itemize}
		\item[-] Set:
		Collection of objects of the same nature (Bag analogy) \\
		Ex: $A = \{1,2,6\}$ 
		\item[-] Empty Set: Set that contains nothing \\
		$ \emptyset = \{\}$
		\item[-] x is an element of a set A \textbf{belongs} \\
		$ x \in A $
		\item[-] If all elements of A are in B \\
		$ A \subseteq B$
		\item[-]  $\mathbb{N}$  The natural numbers \\
		$ \{1, 2, 3, 4, \ldots\} $ \\
		\textit{*Note: Zero is not natural} 
		\item[-]  $\mathbb{Z}$ The Integers \\
		$ \{\ldots, -3, -2, -1, 0, 1, 2, 3, \ldots\}  $
		\item[-] Using the \code{$\ldots$} from the previous example is called \code{Definition by Extension} \\
		\item[-] Two Sets are Equal if: \\
		$A \subseteq B \land B \subseteq A$ \\
		$A=B$
		\item[-] Repetitions and order does not matter in sets.
		\item[-] Sets can also be defined by \code{Comprehension} \\
		Ex: \\ $A = \{x\in S | x \text{ satisfies a property}\}$ \\
		| means \code{such that}\\
		\item[-] Set of Even Numbers \\ 
		$E = \{x \in \mathbb{Z} | n = 2k \text{ for some } k \in \mathbb{Z} \}$	
		\item[-] Set of Odd Numbers \\
		$O = \{x \in \mathbb{Z} | n = 2k+1 \text{ for some } k \in \mathbb{Z} \}$	
		\item[-] Set of Rational Numbers\\
		$\mathbb{Q} = \{ \frac{a}{b} | a \in \mathbb{Z}, \, b \in \mathbb{N}, \, GCD(a,b) =1 \}$	
		\item[-] Cardinality of a set is the \code{number of elements} \\
		 $A = \{1,2,3\},  |A| = 3$ \\
		 $|\{ \{1,2\} \}| = 1 $
		 
		 \begin{center}
		 	\textit{ \textbf{Russel's Set Paradox}} \\
		 	 \end{center}
		 	$$R = \{x \text{ is a set} | x \notin x\}$$ \\
		 	Ex: If $x = \{1, 2, x\}$ then $x\in x$ so $ x \notin R$
		 	
		 	The Question is $R \in R$ ?
		 	\item[1.] If $R \in R$ then $R \notin R$
		 	\item[2.] If $R \notin R$ then $R \in R$ \\
		 	\\ One way to stop the paradox is to forbid recursive sets. \\
		 		\end{itemize}		
	 		
		 	\begin{flushleft}
		 		\underline{\textbf{Operations on Sets}}
		 	\end{flushleft}
		 	
		 	\begin{itemize}
		 		\item[-] We are always assuming a universe $u$. Which is deduced from context. A set that contains all sets which are discussed as subsets.
		 		\item[-] \textbf{Union:} \\ 
		 		$A \cup B = \{e \in u\mid e \in A \textbf{ or } e \in B\}$
		 		\item[-] \textbf{Intersection:} \\ 
		 		$A \cap B =  \{e \in u\mid e \in A \textbf{ and } e \in B\}$
		 		\item[-] \textbf{Complement:} \\ 
		 		$\bar{A} =  \{e \in u\mid e \notin A\}$ 
		 		\item[-] \textbf{Set Difference:}\\
		 		$A \setminus B = A - B$ \\
		 		$\{x \in u \mid x \in A, x \notin B \}$
		 		\item[-] \textbf{Symmetric Difference:} \\
		 		$A \oplus B = \{x \in u \mid x \in A \text{ or } x \in B \} $ \\
		 		
		 		\item[-] Two Identities: \\
		 		$A \oplus B = (A\setminus B) \cup (B\setminus A)$ 
		 	
		 		$A \setminus B = A \cap \bar{B}$ \\
		 	\end{itemize}
	 	
	 	\begin{flushleft}
	 		\underline{\textbf{Laws of Boolean Algebra}}
	 	\end{flushleft}
	 	
	 	\begin{itemize}
	 		\item[-] Identity Laws: \\
	 		$A \cap u = A$,
	 		$A \cup \emptyset = A$
	 		\item[-] Domination Laws: \\
	 		$A \cup u = u$, $A \cap \emptyset = \emptyset$
	 		\item[-] Idempotent Laws: \\
	 		$A \cup A  = A $, $A \cap A = A$
	 		\item[-] Double Complementary Law: \\
	 		$\bar{\bar{A}} = A $
	 		\item[-] Commutative Laws: \\
	 		$A \cup B = B \cup A$, $A \cap B = B \cap A$
	 		\item[-] Associative laws: \\
	 		$A \cup (B \cup C) = (A \cup B) \cup C$ \\
	 		$A \cap (B \cap C) = (A \cap B) \cap C$
	 		\item[-] Distributive Law: \\
	 		Applies for both union and intersection \\
	 		$A \cup (B \cap C) = (A \cup B) \cap (A \cup B)$ \\
	 		 $A \cap (B\cup C) = (A \cap B) \cup (A \cap C)$
	 		\item[-]De Morgan's Laws \\
	 		$(A \cup B)' = \bar{A} \cap \bar{B}$ \\ 
	 		$(A \cap B)'= \bar{A} \cup \bar{B}$
	 		\item[-] Absorption Laws \\
	 		$A \cup (A \cap B) = A $ \\
	 		$A \cap (A \cup B) = A $
	 		\item[-] Set Product \\
	 		$A \times B = \{(a,b) \in R^2 \mid a \in A, b \in B\}$ \\
	 		$|A \times B| = |A| \times |B| $
	 		
	 		\item[-]Power Set \\
	 		$P(A) = $ set of all subsets of $A$ \\
	 		$= \{x \in u \mid x \subseteq A \}$ \\
	 		$|P(A)| = 2^{|A|}$\\
	 		$P(\{1,2,3,\} = \{\emptyset, \{1\}, \{2\}, \{3\} , \{1,2\} ,\{2,3\} , \{1,3\}  ,\{1,2,3\}   \})$
	 		
	 		Each element can be a Yes or No if it shows up in a given combination yes or no is a set of 2 which means it is $2^n$ combinations because you multiply 2 possiblilies n times.
	 		
 			\end{itemize}
		
		 
		 


	
	\paragraph{1.1}
	\textit{
		\textbf{A}	Find this this this and that!
	}
\end{document}
